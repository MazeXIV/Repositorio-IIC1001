\documentclass[12pt]{article}
\usepackage[utf8]{inputenc}
\usepackage[spanish]{babel}
\usepackage{amsmath}
\usepackage{geometry}
\geometry{margin=2.5cm}

\title{Informe sobre el Algoritmo de Euclides}
\author{}
\date{}

\begin{document}

\maketitle

\section*{Introducción}
En este precioso informe se presenta el algoritmo de Euclides, uno de los algoritmos más antiguos y fascinantes en la historia de las matemáticas y la informática. Este algoritmo permite calcular el máximo común divisor de dos números enteros.

\section*{Origen del algoritmo}
El algoritmo fue creado por el matemático griego \textbf{Euclides}, quien lo incluyó en su obra \emph{Los Elementos} alrededor del año 300 a.C.

\section*{Fuente del código}
El código presentado ha sido escrito en el lenguaje de programación Python y sigue directamente la lógica del algoritmo tal como se describe en textos clásicos de teoría de números. Su implementación ha sido importantisima en todo el mundo.

\section*{Utilidad}
El algoritmo de Euclides se utiliza para:
\begin{itemize}
    \item Calcular el máximo común divisor entre dos números.
    \item Simplificar fracciones.
    \item Resolver problemas en criptografía, como en el algoritmo RSA.
    \item Determinar la coprimalidad de dos números.
\end{itemize}

Además, su eficiencia y simplicidad lo hacen ideal para aplicaciones prácticas y educativas.

\section*{Conclusión}
El algoritmo de Euclides es un ejemplo clásico de un algoritmo eficiente y elegante, con una relevancia continua en el estudio de la aritmética y la teoría computacional. Su longevidad y aplicación práctica lo convierten en una herramienta esencial en matemáticas y ciencias de la computación.

\end{document}
